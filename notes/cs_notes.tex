\documentclass[a4paper,12pt]{article}
\usepackage{pstricks}
\usepackage{pst-node}
\usepackage{redefine-sections}
\usepackage[dvips]{geometry}
\geometry{nohead,margin=2cm}

\newcommand{\poslen}{r_\mathrm{0}}
\newcommand{\dislen}{r_\mathrm{D}}
\newcommand{\posvec}{\bar{r}_\mathrm{0}}
\newcommand{\disvec}{\bar{r}_\mathrm{D}}
\newcommand{\nexvec}{\bar{r}_\mathrm{next}}

\begin{document}
\begin{centering}
\noindent{\sf\huge Scattering of light by circum-stellar dust}\\
\end{centering}
\hspace{2ex}

\section*{Scattering, absorption and extinction}
\emph{Based on Ariel Goobar's notes. See
  also~\cite{2008ApJ...686L.103G}}.\hfill\\[2ex]
In order to simulate photons propagating through dust we need
expressions for the mean free paths for \emph{scattering}, $\lambda_s$
and \emph{absorption}, $\lambda_a$, in terms of quantities that we wish
to put into the simulation such as e.g. the observed colour excess,
$E(B-V)$ for the $B-V$ colour. The colour excess is defined as the
difference between the observed, $(B-V)_o$, and emitted colour,
$(B-V)_e$, of an object,
\begin{equation}
E(B-V) = (B-V)_o - (B-V)_e =
-\underbrace{\frac{2.5}{\ln10}}_\varepsilon\left(\ln\frac{f_o^B}{f_o^V}
  -\ln\frac{f_e^B}{f_e^V}\right) = 
  -\varepsilon\ln\left(\frac{f_o^B}{f_e^B}\cdot\frac{f_e^V}{f_o^V}\right)\,,
\label{eq:excess}
\end{equation}
where $f_e$ and $f_o$ are the emitted and observed fluxes of the
object respectively. The observed flux is the difference between the
emitted and extincted flux, $f_r$, which can be calculated over a
distance $r$ for the extinction mean free path $\lambda$ as
\[
f_r = f_e\int_0^r\frac{1}{\lambda}e^{-x/\lambda}\, dx =
f_e\left(1-e^{-r/\lambda}\right)\, .
\]
giving
\begin{eqnarray*}
  f_o^B & = & f_e^B - f_r^B =
    f_e^B\left(1 - 1 +e^{-r/\lambda_B}\right) = f_e^Be^{-r/\lambda_B}\,,\\
  f_o^V & = & f_e^V - f_r^V = f_e^Ve^{-r/\lambda_V}\,.
\end{eqnarray*}
Inserting these into equation~(\ref{eq:excess}), yields
\begin{equation}
E(B-V) = -\varepsilon\ln\left[\exp\left(-\frac{r}{\lambda_B} + 
    \frac{r}{\lambda_V}\right)\right] =
\varepsilon\left(\frac{r}{\lambda_B} - \frac{r}{\lambda_V}\right) =
\varepsilon rn\left(\sigma_B - \sigma_V\right)\,,
\label{eq:excess2}
\end{equation}
where the relation $\lambda = (n\sigma)^{-1}$ between the mean free
path, the number density, $n$, and the cross-section, $\sigma$, was
used in the last step.

Bruce Draine
provides\footnote{\texttt{http://www.astro.princeton.edu/~draine/dust/dust.diel.html}} tables for $\alpha =
\sigma_a/M$ and the albedo, $a$,
\begin{equation}
  a = \frac{\sigma_s}{\sigma_a + \sigma_s}
  \quad\Rightarrow\quad
  \sigma_s = \sigma_a\left(\frac{1}{a} - 1\right)^{-1}
  \quad\Rightarrow\quad
  \lambda_s = \left(\frac{1}{a} - 1\right)\lambda_a
  \,,
  \label{eq:sigmas}
\end{equation}
for different dust environments. Here, $\sigma_s$ and $\sigma_a$ are
the scattering and absorptions cross-sections respectively and $M$ is
the molar mass of the dust particles.

The last relation allows us to calculate the mean free path for
scattering from the absorption length, but an expression for the
latter is still needed. Using these relations the extinction
cross-section, $\sigma$, which is the sum of both scattering and
absorption, can be calculated as
\[
\sigma = \sigma_a + \sigma_s = \frac{\sigma_a}{1 - a} =
\frac{\alpha M}{1-a}\,.
\]
The difference, $\sigma_B-\sigma_V$, between extinction cross section
can then be expressed as
\[
\sigma_B-\sigma_V = M\left(\frac{\alpha_B}{1-a_B} -
  \frac{\alpha_V}{1-a_V}\right)\,,
\]
which together with equation~(\ref{eq:excess2}) gives the number
density as
\[
n = \frac{E(B-V)}{\varepsilon r\left(\sigma_B-\sigma_V\right)} =
\frac{E(B-V)}{Mr\varepsilon}
\left(\frac{\alpha_B}{1-a_B} -
  \frac{\alpha_V}{1-a_V}\right)^{-1} \, ,
\]
This relation can now be used to express the mean free path for
absorption as
\[
\lambda_a = \frac{1}{n\sigma_a} = \frac{1}{nM\alpha} =
\frac{rM\varepsilon}{E(B-V)M\alpha} \left(\frac{\alpha_B}{1-a_B} -
  \frac{\alpha_V}{1-a_V}\right) =
\frac{\varepsilon}{\alpha}\left(\frac{\alpha_B}{1-a_B} -
    \frac{\alpha_V}{1-a_V}\right)\frac{r}{E(B-V)}\, .
\]
The interpretation of the last factor might require some
clarification. Here, $r$, is the thickness of a homogeneous dust layer
and $E(B-V)$ is the colour excess due to \emph{extinction} (i.e.
neither absorbed nor scattered photons will reach the observer) of a
light source observed through the dust layer.

For a circum-stellar dust the colour excess is different, since in
this scenario also scattered photons will be observed.


\section*{A spherical dust shell}
A simple model of a spherical dust-shell with radius, $r$, and inner
radius, $r_i = R\cdot r$, where $0<R<1$ is illustrated in
Figure~\ref{fig:sphere}. Also shown in the figure is a photon leaving
the sphere (solid line).
\begin{figure}[ht]
  \centering
  \psset{unit=.7mm}
\pspicture(-120,-30)(100,90)
\psarc[fillstyle=solid,fillcolor=lightgray]{-}{80.0}{10.0}{190.0}
\psarc[fillstyle=solid,fillcolor=white]{-}{52.0}{9.0}{191.0}
\psline[linestyle=dashed]{->}(0.,0.)(65.00,30.00)
\psline[linewidth=1.0pt,linecolor=black]{->}(65.00,30.00)(-105.00,60.00)
\pcline[linestyle=none](0.0,0.0)(65.0,30.0)
\bput{:U}{$\posvec = (x,y,z)$}
\pcline[linestyle=none](-105.0,60.0)(65.0,30.0)
\aput{:U}{$\disvec$}
\rput[c](65.0,30.0){  \SpecialCoor\psarc{-}{10}{169.992}{204.775}
  \rput[c](14.000;187.384){$\gamma$}}
\psline[linestyle=dashed](0.00,0.00)(38.80,34.62)
\pcline[linestyle=none](0.00,0.00)(38.80,34.62)
\aput{:U}{$r_i$}
\pcline[linestyle=none](7.10,40.22)(38.80,34.62)
\aput{:U}{$q$}
\psline[linestyle=dashed](0.,0.)(7.10,40.22)
\pcline[linestyle=none](0.,0.)(7.10,40.22)
\aput{:U}{$i$}
\rput[c]{169.992}(7.10,40.22){\psline(0.00,4.00)(-4.00,4.00)(-4.00,0.00)}
\rput[c](-60.646,52.173){  \SpecialCoor\psarc{-}{15}{319.295}{349.992}
  \rput[c](18.000;334.644){$\theta$}}
\rput[c]{154.64}(0.,0.){
  \SpecialCoor
  \psline[linestyle=dashed](0.;0.)(80.00;-15.348)
  \pcline[linestyle=none](80.00;-15.35)(0.;0.)
  \aput{:U}{$r$}
  \psline[linestyle=dashed](0.;0.)(80.00;15.348)
  \psarc{-}{30}{-15.35}{0.00}
  \rput[c]{180}(35;-7.674){$\frac{\theta}{2}$}
  \psline[linestyle=dashed](0.,0.)(77.15;0.000)
  \psline[linestyle=dashed](80.00;-15.35)(80.00;15.35)
  \pcline[linestyle=none](80.00;15.35)(80.00;-15.348)
  \bput*{:R}{$l$}
 \rput{-166.85}(80.00;15.35){
  \SpecialCoor
  \psline[linewidth=2pt,linestyle=solid]{-}(30;-90)(65.;90.)
  \pcline[linestyle=none](30;-90)(65.;90.)
  \aput{:R}{$P$}}
 \rput{90}(80.00;-15.35){
   \psarc{-}{7}{-74.65}{74.65}
   \psline(6.00;-37.33)(8.00;-37.33)
   \psline(6.00;37.33)(8.00;37.33)
   \rput{-244.64}(12.00;37.33){$\varphi$}}}
\pcline[linestyle=none](-71.69,54.12)(-60.65,52.17)
\aput{:U}{$d$}
\endpspicture

  \caption{Illustration of a scattered photon (\emph{thin solid line})
    propagating through dust shell. The figure shows the intersecting
    plane of the sphere that is defined by the two vectors $\posvec$
    and $\disvec$.\label{fig:sphere}}
\end{figure}
The current position of the photon is defined by the vector, $\posvec
= (x,y,z)$, and the next position, $\nexvec$, is given by $\nexvec =
\posvec+\disvec$, where $\disvec$ is the displacement vector $\disvec
= (dx,dy,dz)$.

\paragraph{Photon travelling inside the shell} The impact parameter,
$i$, is the minimum distance to the centre of the sphere for any given
photon-path, and can be defined as.
\[
i = \left\{%
  \begin{array}{ll}
    \poslen\cdot\sin\gamma & \textrm{if } 0 \leq \gamma \leq \frac{\pi}{2}\,,\\
    \poslen                & \textrm{if } \frac{\pi}{2} < \gamma\,.
  \end{array} \right.
\]
Here, $\gamma$ is given by
\begin{eqnarray*}
  \posvec\cdot\disvec & = & \poslen\dislen\cos(\pi-\gamma) = 
  x\cdot dx + y\cdot dy + z\cdot dz\\
  \cos\gamma & = & -\cos(\pi-\gamma) = -\frac{x\cdot dx + y\cdot dy + z\cdot dz}{%
    \poslen\dislen}
\end{eqnarray*}
If $i < r_i$, the photon will cross the inner radius of the shell and
then re-enter the shell. This will in turn extend the mean free path
of the photon by the amount $2q = 2\sqrt{r_i^2 + i^2}$, which is the
length of the path within $r_i$.


\paragraph{Path length} For each distance between interactions,
$\dislen$ is added to the total distance travelled by the photon
before leaving the sphere. For a scattered photon leaving the sphere,
the last distance, $s$, travelled inside the sphere is given by
\[
  s^2 = r^2 + \poslen^2 - 2r\poslen\cos\left(\pi-\theta-\gamma\right)\,,
\]
where $\cos\theta = \sin\gamma \cdot \poslen/r$. The distance
travelled by a scattered photon should be compared to the distance
travelled by a non-interacting photon, which means that the distance,
$d$, between the surface of the sphere and the plane (thick, solid
line in Figure~\ref{fig:sphere}) perpendicular to $\disvec$ at
distance $r$ from the centre must be added to the total distance,
where $d$ is given by
\begin{eqnarray*}
  l & = & 2r\cdot\sin\frac{\theta}{2}\\
  \varphi & = & \pi - \frac{\pi}{2} - \frac{\theta}{2} =
  \frac{\pi-\theta}{2}\\
  d & = & l\cdot\cos\varphi =
  2r\cdot\sin\frac{\theta}{2}\cos\frac{\pi-\theta}{2} =
  2r\cdot\sin^2\frac{\theta}{2}\, .
\end{eqnarray*}
The total distance travelled, $D$, can then be summarised as
\[
D = \left\{%
  \begin{array}{ll}
    r & \textrm{for non-interacting photons}\,,\\
    \sum {\dislen}_n + s + d & \textrm{for scattered photons}\,.
  \end{array}\right.
\]

\begin{thebibliography}{99}
  \bibitem{2008ApJ...686L.103G}
    Goobar,~A., \emph{Low $R_{V}$ from Circumstellar Dust around
      Supernova}, ApJ, \textbf{686}, 2008
\end{thebibliography}

\end{document}

